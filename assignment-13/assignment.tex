% Created 2016-04-14 Thu 20:52
\documentclass[11pt]{article}
\usepackage[utf8]{inputenc}
\usepackage[T1]{fontenc}
\usepackage{fixltx2e}
\usepackage{graphicx}
\usepackage{longtable}
\usepackage{float}
\usepackage{wrapfig}
\usepackage{rotating}
\usepackage[normalem]{ulem}
\usepackage{amsmath}
\usepackage{textcomp}
\usepackage{marvosym}
\usepackage{wasysym}
\usepackage{amssymb}
\usepackage{capt-of}
\usepackage[hidelinks]{hyperref}
\tolerance=1000
\usepackage[utf8]{inputenc}
\usepackage{commath}
\usepackage{pgf}
\usepackage{tikz}
\usetikzlibrary{shapes,backgrounds}
\usepackage{marginnote}
\usepackage{listings}
\usepackage{enumerate}
\usepackage{algpseudocode}
\usepackage{algorithm}
\usepackage{mathtools}
\setlength{\parskip}{16pt plus 2pt minus 2pt}
\renewcommand{\arraystretch}{1.6}
\author{Oleg Sivokon}
\date{\textit{<2016-04-09 Sat>}}
\title{Assignment 13, Data-Structures}
\hypersetup{
  pdfkeywords={Data-Structures, Algorithms, Assignment},
  pdfsubject={Third assignment in the course Data-Structures},
  pdfcreator={Emacs 25.1.50.2 (Org mode 8.2.10)}}
\begin{document}

\maketitle
\tableofcontents


\definecolor{codebg}{rgb}{0.96,0.99,0.8}
\definecolor{codestr}{rgb}{0.46,0.09,0.2}
\lstset{%
  backgroundcolor=\color{codebg},
  basicstyle=\ttfamily\scriptsize,
  breakatwhitespace=false,
  breaklines=false,
  captionpos=b,
  framexleftmargin=10pt,
  xleftmargin=10pt,
  framerule=0pt,
  frame=tb,
  keepspaces=true,
  keywordstyle=\color{blue},
  showspaces=false,
  showstringspaces=false,
  showtabs=false,
  stringstyle=\color{codestr},
  tabsize=2
}
\lstnewenvironment{maxima}{%
  \lstset{%
    backgroundcolor=\color{codebg},
    escapeinside={(*@}{@*)},
    aboveskip=20pt,
    captionpos=b,
    label=,
    caption=,
    showstringspaces=false,
    frame=single,
    framerule=0pt,
    basicstyle=\ttfamily\scriptsize,
    columns=fixed}}{}
}
\makeatletter
\newcommand{\verbatimfont}[1]{\renewcommand{\verbatim@font}{\ttfamily#1}}
\makeatother
\verbatimfont{\small}%
\clearpage

\section{Problems}
\label{sec-1}

\subsection{Problem 1}
\label{sec-1-1}
\begin{enumerate}
\item How many comparisons does the \texttt{partition} algorithm have to make before it
completes on an input of size $n$, provides the input is sorted in
ascending order?
\item What if the input is sorted in descending order?
\end{enumerate}

\subsection{Problem 2}
\label{sec-1-2}
Array is said to be almost sorted with error of size $k$ when for any two
elements it holds that if there is more than $k$ elements between them,
then they are in correct order.

\begin{enumerate}
\item Modify \texttt{quicksort} algorithm to produce almost sorted arrays.
\item What is the running time of this new algorithm?
\end{enumerate}

\subsection{Problem 3}
\label{sec-1-3}
Show the steps taken by the algorithm \texttt{randomized-select} on the sequence
$S = (60, 70, 80, 90, 100, 1, 5, 9, 11, 15, 19, 21, 25, 29, 30)$ for $k = 5$.

\subsubsection{Answer 3}
\label{sec-1-3-1}
This algorithm will use the same \texttt{random-partition} procedure used in
\texttt{quicksort}:
\begin{enumerate}
\item $\textit{random-partition}(S, S_1, S_2)$, randomly selected 19, hence:
$S_1 = (1, 5, 9, 11, 15, 19)$ and 
$S_2 = (60, 70, 80, 90, 100, 21, 25, 29, 30)$.
\item Since $\abs{S_1} = 6 > k$ the $k^{th}$ element must be in $S_1$, recure
on it.
\item $\textit{random-partition}(S_1, S_3, S_4)$, randomly select 5, hence:
$S_3 = (1, 5)$ and
$S_4 = (9, 11, 15, 19)$.
\item Since $\abs{S_3} = 2 < k$, decrease $k$ by 2 and recurse on $S_4$:
\item $k = k - 2$.
\item $\textit{random-partition}(S_4, S_5, S_6)$, randomly select 11, hence:
$S_5 = (9, 11)$ and
$S_6 = (15, 19)$.
\item Since $\abs{S_5} = 2 = k$ we are done, the $k^{th}$ element therefore
is 11.
\end{enumerate}

\subsection{Problem 4}
\label{sec-1-4}
\begin{enumerate}
\item Prove that a sequence $S$ of length $n$ has at most three elements that
repeat more than $\lfoor\frac{n}{4}\rfloor$ times.
\item Write an algorithm finding all elements which repeat more than
$\lfoor\frac{n}{4}\rfloor$ times.  Running time $O(n)$.
\end{enumerate}

\subsubsection{Answer 4}
\label{sec-1-4-1}
Assume for contradiction that there are four elements in $S$ that repeat
$\lfoor\frac{n}{4}\rfloor$ times.  That is exist an element in the sequence
that repeats $\lfoor\frac{n}{4} + 1\rfloor$ times.  Note that
$3\lfoor\frac{n}{4}\rfloor + \lfoor\frac{n}{4} + 1\rfloor > n$ contrary to
assumed.  Hence, by contracition, there are at most three such elements.

\subsubsection{Answer 5}
\label{sec-1-4-2}
Simple solution involves using hash-table: loop over the sequence, using
elements as keys in the hash-table.  Increment the key whenever you
encounter the element.  Finally, loop over the hash-table to collect all
those for which the condition holds.

This can be improved by using $\lfoor\frac{n}{4}\rfloor$ hash-tables, first
for the elements which have been seen just once, second---for the elements
seen twice and so on.  Thus, instead of icnrementing the counter, the
elements would be promoted to the last hash-table.  This removes the
requirement of the final loop, but doesn't change the running time
significantly.

A more complicated way to do this, without using hash-tables is to do the
following:
\begin{enumerate}
\item Partition in three parts using the same partition procedure used in
\texttt{quick-sort} algorithm.  If a partition is smaller than
$\lfoor\frac{n}{4}\rfloor$, throw it away and repeat.
\item At this point, the following could happen:
\begin{itemize}
\item You have three partitions each containing the same element---you are
done.
\item Further partitioning is impossible: no such elements exist---you are
done.
\item Some partitions only contain same elements (store the element, throw
away the partition), others contain different element: throw them away
(they aren't candidates).
\end{itemize}
\end{enumerate}

\subsection{Problem 5}
\label{sec-1-5}
\begin{enumerate}
\item Given a sequence of real numbers: $(a_0, a_1, a_2, \dots, a_n)$ define:
\begin{align*}
  m &= \min\{a_i\;|\; 0 \leq i \leq n\} \\
  M &= \max\{a_i\;|\; 0 \leq i \leq n\}
\end{align*}

Show that there exist $a_i, a_j$ s.t.:
\begin{align*}
  \abs{a_i - a_j} &\leq \frac{M - m}{n}
\end{align*}

\item Write the algorithm for finding the $a_i, a_j$ from the question above.
\end{enumerate}

\subsubsection{Answer 5}
\label{sec-1-5-1}
Note that the sequence with the given $M$ and $m$ will have $M - m = k \geq
    n$ elements.  Suppose now we arrange the elements in increasing order.
Suppose, for contradiction, that none of the elements of the sequence
satisfies $\abs{a_i - a_j} \leq \frac{M - m}{n}$, then it also means that
the difference between every two adjacent elements must be at least
$\frac{M - m + \epsilon}{n}$ for some positive $\epsilon$.  since there are
k pairs of adjoint elements, we get that:

\begin{align*}
  \sum_{i=1}^k\left(a_i - a_{i-1}\right) &= \sum_{i=1}^k\left(\frac{M - m + \epsilon}{n}\right) \\
  &= \frac{k}{n}\left(M - m\right) + k \epsilon \geq M - m
\end{align*}

contrary to assumed.  Hence, by contradiction, there must be at least one
pair $a_i, a_j$ s.t. $\abs{a_i - a_j} \leq \frac{M - m}{n}$.

\subsubsection{Answer 6}
\label{sec-1-5-2}
The general idea for the algorithm is to normalize all members of the given
sequence by subtracting the minimum and mutliplying by the ratio of one less
than the length of the sequence and the difference between the maximum and
the minumum.  Once done, do the insertion sort: two elements which fall in
the same cell will have a distance between them less than the one between
the minimum and the maximum divided into one less than the number of elements.

\begin{algorithm}
  \caption{Find $x, y \in Elts$ s.t. $\abs{x - y} \leq \frac{\max(Elts) - \min(Elts)}{\abs{Elts} - 1}$}
  \begin{algorithmic}
    \Procedure {$\textit{min-pair}$}{$elements$}
    \State \Call {$max \leftarrow \texit{max}$}{$elements$}
    \State \Call {$min \leftarrow \texit{min}$}{$elements$}
    \State \Call {$size \leftarrow \texit{size}$}{$elements$}
    \State \Call {$copy \leftarrow \textit{make-vector}$}{$size, nil$}
    \For {$element \in elements$} \Do
    \State {$index \leftarrow \lfloor \frac{(elt - min) \times (size - 1)}{max - min} \rfloor$}
    \If {$copy_{index} = nil$} \Then
    \State {$copy_{index} = element$}
    \Else
    \Return {$element, copy_{index}$}
    \EndIf
    \EndFor
    \EndProcedure
  \end{algorithmic}
\end{algorithm}
% Emacs 25.1.50.2 (Org mode 8.2.10)
\end{document}