% Created 2016-03-19 Sat 23:28
\documentclass[11pt]{article}
\usepackage[utf8]{inputenc}
\usepackage[T1]{fontenc}
\usepackage{fixltx2e}
\usepackage{graphicx}
\usepackage{longtable}
\usepackage{float}
\usepackage{wrapfig}
\usepackage{rotating}
\usepackage[normalem]{ulem}
\usepackage{amsmath}
\usepackage{textcomp}
\usepackage{marvosym}
\usepackage{wasysym}
\usepackage{amssymb}
\usepackage{capt-of}
\usepackage[hidelinks]{hyperref}
\tolerance=1000
\usepackage[utf8]{inputenc}
\usepackage{commath}
\usepackage{pgf}
\usepackage{tikz}
\usetikzlibrary{shapes,backgrounds}
\usepackage{marginnote}
\usepackage{listings}
\usepackage{enumerate}
\usepackage{algpseudocode}
\usepackage{algorithm}
\usepackage{mathtools}
\setlength{\parskip}{16pt plus 2pt minus 2pt}
\renewcommand{\arraystretch}{1.6}
\author{Oleg Sivokon}
\date{\textit{<2016-03-06 Sun>}}
\title{Assignment 11, Data-Structures}
\hypersetup{
 pdfauthor={Oleg Sivokon},
 pdftitle={Assignment 11, Data-Structures},
 pdfkeywords={Data-Structures, Algorithms, Assignment},
 pdfsubject={First assignment in the course Data-Structures},
 pdfcreator={Emacs 25.0.50.1 (Org mode 8.3beta)}, 
 pdflang={English}}
\begin{document}

\maketitle
\tableofcontents

\definecolor{codebg}{rgb}{0.96,0.99,0.8}
\definecolor{codestr}{rgb}{0.46,0.09,0.2}
\lstset{%
  backgroundcolor=\color{codebg},
  basicstyle=\ttfamily\scriptsize,
  breakatwhitespace=false,
  breaklines=false,
  captionpos=b,
  framexleftmargin=10pt,
  xleftmargin=10pt,
  framerule=0pt,
  frame=tb,
  keepspaces=true,
  keywordstyle=\color{blue},
  showspaces=false,
  showstringspaces=false,
  showtabs=false,
  stringstyle=\color{codestr},
  tabsize=2
}
\lstnewenvironment{maxima}{%
  \lstset{%
    backgroundcolor=\color{codebg},
    escapeinside={(*@}{@*)},
    aboveskip=20pt,
    captionpos=b,
    label=,
    caption=,
    showstringspaces=false,
    frame=single,
    framerule=0pt,
    basicstyle=\ttfamily\scriptsize,
    columns=fixed}}{}
}
\makeatletter
\newcommand{\verbatimfont}[1]{\renewcommand{\verbatim@font}{\ttfamily#1}}
\makeatother
\verbatimfont{\small}%
\clearpage

\section{Problems}
\label{sec:orgheadline12}

\subsection{Problem 1}
\label{sec:orgheadline3}
Count the number of compare and copy operations required to sort the two
sequences given below using insertion sort:

\begin{enumerate}
\item \(\frac{n}{2}, \frac{n}{2}-1, \dots, 2, 1, n, n-1, \dots, \frac{n}{2},
      \frac{n}{2}-1\).
\item \(n, 1, n-1, 2, \dots, \frac{n}{2}+2, \frac{n}{2}-1, \frac{n}{2}+1,
      \frac{n}{2}\).
\end{enumerate}

\subsubsection{Answer 1}
\label{sec:orgheadline1}
Assuming we sort in ascending order, observe that the our task is to repeat
the same operation twice (viz. to reverse two sorted arrays of the size
exactly half of \(n\).)  Reversing individual arrays will require in case of
one-length array 0 \texttt{swap} operations, in case of two-length array, 1 \texttt{swap}
operation, and when we go further, we would need to do the same amount of
work we did in the case of \(n-1\), and then need to do \(n-1\) more swaps to
bring the first element of the original input to the back.

This gives the recurrence (for reversing the array):

\begin{align*}
  R(1) &= 0 \\
  R(2) &= 1 \\
  R(3) &= R(2) + (3 - 1) = 3 \\
  R(4) &= R(3) + (4 - 1) = 6 \\
  \dots \\
  R(n) &= R(n-1) + n - 1 = \sum_{i=0}^{n-1}i = \frac{n(n-1)+2}{2}\;.
\end{align*}

Hence the total amount of \texttt{swap} operations needed to sort the given array
is \(T(n) = 2R(\frac{n}{2}) = n(n+1)\).

It is easy to see that the asymptotic complexity of \(T(n)\) is \(O(n^2)\) since
since \(\lim_{n \to \infty}\frac{n^2}{n(n-1)}=1\).

\subsubsection{Answer 2}
\label{sec:orgheadline2}
Assuming we sort in ascending order, in the first step we make one
comparison and swap.  In the next step the first element will stand in its
place, while the second element will need to move two positions to the end.
The one before last element will need to move one position back.

The same will happen once we increment further.  I.e. now two elements will
stand in their place, but now the largest element will need to move two
positions to the end, and so will the second largest element.  The third
smalles element will need to move one position to the front.

Let now \(T(n)\) denote the number of \texttt{swap} operations we perform for any
given \(n\), then:

\begin{align*}
  T(2) &= 1 \\
  T(4) &= T(2) + 2 + 1 \\
  T(6) &= T(4) + 2 + 2 + 1 \\
  T(8) &= T(6) + 2 + 2 + 2 + 1 \\
  \dots \\
  T(n) &= T(n-2) + n - 1\;.
\end{align*}

This recurrence is easily recognizable as being just \((n-1)^2\).  Hence, as
expected, insertion sort requires roughly quadratic number of swaps,
i.e. \(O(n^2)\).

\subsection{Problem 2}
\label{sec:orgheadline5}
Given a sorted array \(A[1\dots n]\) where all elements are unique integers:
\begin{enumerate}
\item Write a predicate that asserts whether the given array is dense
(has no gaps) in time \(\Theta(1)\).
\item Write a predicate that given that \(A\) is sparse finds the element \(v\)
which doesn't apear in \(A\) but is smaller than its largest element
and is greater than its smallest element.
\end{enumerate}

\subsubsection{Answer 3}
\label{sec:orgheadline4}
\begin{algorithm}
  \caption{Assert $S$ is a sparse array}
  \begin{algorithmic}
    \Procedure {$\textit{is-sparse}$}{$S$}
    \State \Call {$x \leftarrow first$}{$S$}
    \State \Call {$y \leftarrow last$}{$S$}
    \State \Call {$len \leftarrow length$}{$S$}
    \State \Return {$y - x > len$}
    \EndProcedure
  \end{algorithmic}
\end{algorithm}

\begin{algorithm}
  \caption{Finds the first gap in sparse array $S$}
  \begin{algorithmic}
    \Procedure {$\textit{binsearch-missing}$}{$S$}
    \State \Call {$len \leftarrow  length$}{$S$}
    \State {$start \leftarrow 0$} {$end \leftarrow len / 2$}
    \State \Call {$cut \leftarrow slice$}{$S, start, end$}
    \While {$end - start > 1$}
    \If {$\textit{is-sparse}(cut)$} \Then
    \State {$end \leftarrow end - (end - start) / 2$}
    \Else
    \State {$start \leftarrow end - 1$}
    \State {$end \leftarrow end + (len - end) / 2$}
    \EndIf
    \EndWhile
    \State \Return {$end + 1$}
    \EndProcedure
  \end{algorithmic}
\end{algorithm}

Real code (compiled in C99):

\emph{Note that you will need the support code located in this derictory}

\begin{verbatim}
bool is_sparse(const array* sorted) {
    int* first = (int*)sorted->elements[0]->val;
    int* last = (int*)sorted->elements[sorted->length - 1]->val;

    return (int)*last - (int)*first >= sorted->length;
}

size_t binsearch_missing(const array* sparse) {
    size_t start = 0, end = sparse->length / 2;
    array* cut = slice(sparse, start, end);

    while (end - start > 1) {
        if (is_sparse(cut)) {
            end -= (end - start) / 2;
        } else {
            start = end - 1;
            end += (sparse->length - end) / 2;
        }
        free_array(cut);
        cut = slice(sparse, start, end);
    }
    return end + 1;
}

void report(array* tested, char* message) {
    printf(message, to_string((printable*)tested));
    if (!is_sparse(tested)) {
        printf("Array is dense.\n");
    } else {
        printf("Array is sparse.\n");
        size_t missing = binsearch_missing(tested);
        printf("The first gap is at: %d\n", (int)missing);
    }
}

int main() {
    report(make_sparse_sorted_array(
        10, 13, 7, int_element_generator),
           "Created sparse array: %s.\n");
    return 0;
}
\end{verbatim}

\begin{verbatim}
Created sparse array: [13, 17, 18, 24, 30, 31, 35, 41, 45, 49].
Array is sparse.
The first gap is at: 2
\end{verbatim}

\subsection{Problem 3}
\label{sec:orgheadline7}
Given a list of \(m\) real numbers \(S\), a similar list of \(n\) real numbers \(T\)
and a real number \(z\), write an algorithm that finds a pair of elements in
\(x \in S\) and \(t \in T\) s.t. \(s + t = z\).

\subsubsection{Answer 4}
\label{sec:orgheadline6}
\begin{algorithm}
  \caption{Find $s \in S$ and $t \in T$ s.t. $s + t = z$}
  \begin{algorithmic}
    \Procedure {$\textit{summands-of}$}{$S, T, z$}
    \If {$length(S) < length(T)$} \Then
    \State \Call {$shortest \leftarrow sorted$}{$S$}
    \State {$longest \leftarrow T$}
    \Else
    \State \Call {$shortest \leftarrow sorted$}{$T$}
    \State {$longest \leftarrow S$}
    \EndIf
    \For {$val \in longest$}
    \State {$diff \leftarrow z - val$}
    \State \Call {$(pos, found) \leftarrow binsearch$}{$shortest, diff$}
    \If {$found$} \Then
    \State \Call {$other \leftarrow elt$}{$shortest, pos$}
    \State \Return {$(val, other)$}
    \EndIf
    \EndFor
    \State \Return {$failure$}
    \EndProcedure
  \end{algorithmic}
\end{algorithm}

Real code compiled in C99:

\begin{verbatim}
pair* summands_of(const array* a,
                  const array* b,
                  const float z,
                  comparison_fn_t cmp) {
    pair* result = make_pair();
    array* shortest;
    array* longest;
    size_t i;

    if (a->length < b->length) {
        shortest = sorted((array*)a, cmp);
        longest = (array*)b;
    } else {
        shortest = sorted((array*)b, cmp);
        longest = (array*)a;
    }
    for (i = 0; i < longest->length; i++) {
        float* val = longest->elements[i]->val;
        printable_float* diff = make_printable_float(z - *val);
        size_t pos = binsearch(shortest, (printable*)diff, cmp);
        if (pos >= shortest->length) continue;
        float* other = shortest->elements[pos]->val;
        result->first =
            (printable*)make_printable_float((float)*val);
        result->last =
            (printable*)make_printable_float((float)*other);
            break;
    }
    return result;
}

int main() {
    int ints[7] = {1, 2, 3, 4, 5, 6, 7};
    float sum = 13.0;
    array* test = make_array_from_pointer(
        ints, 7, float_element_generator);

    printf("Floats: %s\n", to_string((printable*)test));
    pair* summands = summands_of(test, test, sum, compare_floats);
    printf("%f = %s + %s\n",
           sum,
           to_string(summands->first),
           to_string(summands->last));
    return 0;
}
\end{verbatim}

\begin{verbatim}
Floats: [1.000000, 2.000000, 3.000000, 4.000000, 5.000000, 6.000000, 7.000000]
13.000000 = 7.000000 + 6.000000
\end{verbatim}


\subsection{Problem 4}
\label{sec:orgheadline9}
Show example of a function \(f\) satisfying \(f(n) \neq \Omega(n)\) and
\(f(n) \neq O(n)\).

\subsubsection{Anwser 5}
\label{sec:orgheadline8}
Recall the definition of \(O(n)\): \(f(n) = O(f(n))\) as \(n \to \infty\)
precisely when \(\forall (x \geq x_0): \abs{f(n)} \leq M \abs{f(n)}\), where
\(M\) and \(x_0\) are some real numbers.  The definition of \(\Omega\) is similar,
but asking to find a real constant \(M\) s.t. starting with \(x_0\) all values
of \(\abs{f(n)}\) are less than \(M\abs{f(n)}\).

One way to come up with the function which isn't its own upper or lower bound
is to take an oscilating function, for example:

\begin{align*}
  f(n) &= \begin{cases}
    1, &\textbf{if}\; n \equiv 0 \mod 2 \\
    0, &\textbf{if}\; n \equiv 1 \mod 2
  \end{cases}
\end{align*}

Clearly there is no such \(n_0\) for which all values of \(f(n)\) are greater
than \(f(n_0)\), similarly, there are no such \(n_0\) that all values of \(f(n)\)
for \(n > n_0\) are smaller than any multiple of \(\abs{f(n)}\).

\subsection{Problem 5}
\label{sec:orgheadline11}
Given following functions:

\begin{align*}
  f_1(n) &= max\left(\sqrt{n^3} \times \lg n, \sqrt[3]{n^4} \times \lg^5 n\right) \\
  f_2(n) &= \begin{cases}
    n \times \lg^3 n, &\textbf{if}\; n = 2k \\
    n^3 \times \lg^3 n, &\textbf{if}\; n = 2k + 1
  \end{cases} \\
  f_3(n) &= n^{\lg\lg n} + n^{1000000} \times \lg^{100000} n \\
  f_4(n) &= \begin{cases}
    n^n \times 2^{n!}, &\textbf{if}\; n \leq 2^{1000000} \\
    \sqrt{n^{\lg n}}, &\textbf{if}\; n > 2^{1000000}
  \end{cases}
\end{align*}

for each pair of them assert \(O\), \(o\), \(\Omega\), \(\omega\) and \(\Theta\).

\subsubsection{Answer 6}
\label{sec:orgheadline10}
\begin{center}
\begin{tabular}{l|llllll}
 & \(f_1, f_2\) & \(f_1,f_3\) & \(f_1,f_4\) & \(f_2,f_3\) & \(f_2,f_4\) & \(f_3,f_4\)\\
\hline
\(O\) & \(f_1 = O(f_2)\) & \(f_1 = O(f_3)\) & \(f_1 = O(f_4)\) & \(f_2 = O(f_3)\) & \(f_2 = O(f_4)\) & \(f_3 = O(f_4)\)\\
\(o\) & \(f_1 = o(f_2)\) & \(f_1 = o(f_3)\) & \(f_1 = o(f_4)\) & \(f_2 = o(f_3)\) & \(f_2 = o(f_4)\) & \(f_3 = o(f_4)\)\\
\(\Omega\) & \(f_2 = \Omega(f_1)\) & \(f_3 = \Omega(f_1)\) & \(f_4 = \Omega(f_1)\) & \(f_3 = \Omega(f_2)\) & \(f_4 = \Omega(f_2)\) & \(f_4 = \Omega(f_3)\)\\
\(\omega\) & \(f_2 = \omega(f_1)\) & \(f_3 = \omega(f_1)\) & \(f_4 = \omega(f_1)\) & \(f_3 = \omega(f_2)\) & \(f_4 = \omega(f_2)\) & \(f_4 = \omega(f_3)\)\\
\(\Theta\) & \(f_1 \neq \Theta(f_2)\) & \(f_1 \neq \Theta(f_3)\) & \(f_1 \neq \Theta(f_4)\) & \(f_2 \neq \Theta(f_3)\) & \(f_2 \neq \Theta(f_4)\) & \(f_3 \neq \Theta(f_4)\)\\
\end{tabular}
\end{center}

\textbf{Discussion}

We can simplify the calculations by noticing that \(f_1\) is sub-linear,
i.e. it is dominated by \(O(n)\), while all other functions are at least
linear.

We are only interested in the second case of \(f_4\), which is easier to
rewrite as \(n^{\frac{1}{2}\lg n}\).

Similarly, we can simplify \(f_3\) by taking its fastest growing term: \(n^{\lg
    \lg n}\).  In other words, both \(f_3\) and \(f_4\) exhibit exponential growth.

Finally, \(f_2\) is qubic in its worst case (again, the logarithmic factor is
dominated by \(n^3\) asymptotically.)

We can see that \(f_4\) grows faster than \(f_3\) by comparing the exponents:
\(\frac{1}{2}\lg n > \lg \lg n\) for some \(n\) since \(\frac{1}{2}\lg n = \lg
    \sqrt{n} > \lg \lg n\) since \(\lg n = O(\sqrt{n})\).

Neither function is of the same order as the other, thus it is never the
case that \(f_i = \Theta(f_j)\).
\end{document}