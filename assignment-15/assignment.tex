% Created 2016-05-14 Sat 22:51
\documentclass[11pt]{article}
\usepackage[utf8]{inputenc}
\usepackage[T1]{fontenc}
\usepackage{fixltx2e}
\usepackage{graphicx}
\usepackage{longtable}
\usepackage{float}
\usepackage{wrapfig}
\usepackage{rotating}
\usepackage[normalem]{ulem}
\usepackage{amsmath}
\usepackage{textcomp}
\usepackage{marvosym}
\usepackage{wasysym}
\usepackage{amssymb}
\usepackage{capt-of}
\usepackage[hidelinks]{hyperref}
\tolerance=1000
\usepackage[utf8]{inputenc}
\usepackage{commath}
\usepackage{pgf}
\usepackage{tikz}
\usetikzlibrary{shapes, arrows}
\usepackage{marginnote}
\usepackage{listings}
\usepackage{enumerate}
\usepackage{algpseudocode}
\usepackage{algorithm}
\usepackage{mathtools}
\setlength{\parskip}{16pt plus 2pt minus 2pt}
\renewcommand{\arraystretch}{1.6}
\author{Oleg Sivokon}
\date{\textit{<2016-05-14 Sat>}}
\title{Assignment 15, Data-Structures}
\hypersetup{
  pdfkeywords={Data-Structures, Algorithms, Assignment},
  pdfsubject={Third assignment in the course Data-Structures},
  pdfcreator={Emacs 25.1.50.2 (Org mode 8.2.10)}}
\begin{document}

\maketitle
\tableofcontents


\definecolor{codebg}{rgb}{0.96,0.99,0.8}
\definecolor{codestr}{rgb}{0.46,0.09,0.2}
\lstset{%
  backgroundcolor=\color{codebg},
  basicstyle=\ttfamily\scriptsize,
  breakatwhitespace=false,
  breaklines=false,
  captionpos=b,
  framexleftmargin=10pt,
  xleftmargin=10pt,
  framerule=0pt,
  frame=tb,
  keepspaces=true,
  keywordstyle=\color{blue},
  showspaces=false,
  showstringspaces=false,
  showtabs=false,
  stringstyle=\color{codestr},
  tabsize=2
}
\lstnewenvironment{maxima}{%
  \lstset{%
    backgroundcolor=\color{codebg},
    escapeinside={(*@}{@*)},
    aboveskip=20pt,
    captionpos=b,
    label=,
    caption=,
    showstringspaces=false,
    frame=single,
    framerule=0pt,
    basicstyle=\ttfamily\scriptsize,
    columns=fixed}}{}
}
\makeatletter
\newcommand{\verbatimfont}[1]{\renewcommand{\verbatim@font}{\ttfamily#1}}
\makeatother
\verbatimfont{\small}%
\clearpage

\section{Problems}
\label{sec-1}

\subsection{Problem 1}
\label{sec-1-1}
\begin{enumerate}
\item How many comparisons does the \texttt{quicksort} algorithm do on the input of
size 3 in the best, worst and average cases?
\item Graph the decision tree for the above question, comment on how it
represent the answers to the previous question.
\item How many comparisions does the \texttt{heapsort} do on the input of size 4
in the best, worst and average cases?
\item Show that for the input size 4, \texttt{heapsort} is sub-optimal.  Explain why
this doesn't contradict general optimality claims.
\end{enumerate}

\subsubsection{Answer 1}
\label{sec-1-1-1}
\tikzstyle{decision} = [diamond, draw, text width=4.5em, text badly centered, node distance=3cm, inner sep=0pt]
\tikzstyle{block} = [rectangle, draw, text centered]

\begin{tikzpicture}[->,>=stealth',shorten >=1pt,auto,node distance=3cm, semithick]
  % Place nodes
  \node [block]                     (init)   {input array: $(a,b,c)$};
  \node [decision, below of=init]   (part11) {$a < c$};
  \node [decision, left of=part11]    (part21) {$b < c$};
  \node [decision, right of=part11]    (part22) {$b < c$};
  \node [block, below of=part21] (b11) {$(a,c,b)$};
  \node [block, below of=part22] (b12) {$(a,c,b)$};
  \node [decision, left of=b11] (part31) {$a < b$};
  \node [decision, left of=b12] (part32) {$b < a$};
  \node [block, below of=part31] (b21) {$(a,b,c)$};
  \node [block, right of=b21] (b22) {$(b,a,c)$};
  \node [block, below of=part32] (b23) {$(c,b,a)$};
  \node [block, right of=b23] (b24) {$(c,a,b)$};

  % Draw edges
  \path (init)   edge node {partition around $c$} (part11)
        (part11) edge node {$(a,b,c)$} (part21)
        (part11) edge node {$(c,b,a)$} (part22)
        (part21) edge node {$(a,b,c)$} (part31)
        (part21) edge node {pivot is in the middle} (b11)
        (part22) edge node {pivot is in the middle} (b12)
        (part22) edge node {$(c,b,a)$} (part32)
        (part31) edge node {} (b21)
        (part31) edge node {} (b22)
        (part32) edge node {} (b23)
        (part32) edge node {} (b24);
\end{tikzpicture}

Which also answers the question of number of comparison that need to be made:
\begin{enumerate}
\item Best case scenario: 2 comparisons.  If we are lucky, the pivot element ends
up in the middle of the partitioned array, thus all resulting sub-arrays
are of size 1 and need not be partitioned further.
\item Worst case scenario: 3 comparisons.  Ohterwise, we'll need to compare both
other elements with the pivot and then break the tie between the remaining
elements.
\item As you can see above, we are ``lucky'' only 2 times out of 6.  Taking
weighted average gives us $2 \times 2 \times \frac{1}{6} + 3 \times 4
       \times \frac{1}{6} = \frac{8}{3}$.
\end{enumerate}

\subsection{Problem 2}
\label{sec-1-2}
Design data-structure containing two independent queues both using the same
``circular array'' for storage.  Define necessary operations: \emph{insertion},
\emph{deletion}, \emph{boundary-checking}.

\subsubsection{Answer 2}
\label{sec-1-2-1}
The idea is exactly the same as it was for the single queue, however in this
case the elements of the first queue will be positioned at odd indices, while
elements of the second queue will be positioned on the even indices.  We will
also need to keep four variables storing the position of the head and the tail
of each of the two queues. \emph{(See figure on the next page.)}

\begin{algorithm}
  \caption{Double FIFO queue}
  \begin{algorithmic}
    \Procedure {$\textit{push}$}{$element, queue$}
    \If {$\textit{can-push}(queue)$}
    \State \Call {$size \leftarrow \textit{size}$}{$queue$}
    \State \Call {$tail \leftarrow \textit{tail}$}{$queue$}
    \State \Call {$head \leftarrow \textit{head}$}{$queue$}
    \If {$\textit{is-even}(queue)$}
    \State {$queue[tail \times 2] \leftarrow element$}
    \Else
    \State {$queue[tail \times 2 + 1] \leftarrow element$}
    \EndIf
    \State {$tail \leftarrow (tail + 1) \mod size$}
    \EndIf
    \EndProcedure

    \Procedure {$\textit{pop}$}{$queue$}
    \If {$\textit{can-pop}(queue)$}
    \State \Call {$size \leftarrow \textit{size}$}{$queue$}
    \State \Call {$tail \leftarrow \textit{tail}$}{$queue$}
    \State \Call {$head \leftarrow \textit{head}$}{$queue$}
    \If {$\textit{is-even}(queue)$}
    \State {$index \leftarrow head \times 2$}
    \Else
    \State {$index \leftarrow head \times 2 + 1$}
    \EndIf
    \State {$head \leftarrow (head + 1) \mod size$}
    \State \Return {$result$}
    \EndIf
    \EndProcedure

    \Procedure {$\textit{can-push}$}{$queue$}
    \State \Call {$size \leftarrow \textit{size}$}{$queue$}
    \State \Call {$tail \leftarrow \textit{tail}$}{$queue$}
    \State \Call {$head \leftarrow \textit{head}$}{$queue$}
    \State \Return {$(tail + 1) \mod size \neq head$}
    \EndProcedure

    \Procedure {$\textit{can-pop}$}{$queue$}
    \State \Call {$tail \leftarrow \textit{tail}$}{$queue$}
    \State \Call {$head \leftarrow \textit{head}$}{$queue$}
    \State \Return {$tail \neq head$}
    \EndProcedure
  \end{algorithmic}
\end{algorithm}

\subsection{Problem 3}
\label{sec-1-3}
Given set $S$ s.t. $S \subset \mathbb{N}$, $\abs{S} = n$, $\max(S) = n^k-1, k
   \geq 0$.  Also given natural number $z$.
\begin{enumerate}
\item Write an algorithm for finding two distinct summands of $z$ in $S$,
s.t. its running time is $\Theta(n \times \min(k, \lg n))$.
\item Same as above, but find three distinct summands.  Running time
$\Theta(n^2)$.
\item Same as above, but for four distinct summands.  Running time $\Theta(n^2
      \times \min(k, \lg n))$.
\item Same as above, but for five distinct summands.  Running time
$\Theta(n^3)$.
\end{enumerate}

\subsubsection{Answer 3}
\label{sec-1-3-1}

\subsection{Problem 4}
\label{sec-1-4}
Given list of points $P = \{(x, y) \;|\; x^2 + y^2 \leq 0, x \geq 0\}$,
assuming uniform random distribution of points across the semi-circle, write
an algorithm for sorting them on $\tan \theta$, where $\theta$ is the angle
between $x$ axis and the line through origin and the given point.

\subsubsection{Answer 4}
\label{sec-1-4-1}
% Emacs 25.1.50.2 (Org mode 8.2.10)
\end{document}