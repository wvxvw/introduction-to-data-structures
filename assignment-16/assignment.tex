% Created 2016-06-11 Sat 18:44
\documentclass[11pt]{article}
\usepackage[utf8]{inputenc}
\usepackage[T1]{fontenc}
\usepackage{fixltx2e}
\usepackage{graphicx}
\usepackage{longtable}
\usepackage{float}
\usepackage{wrapfig}
\usepackage{rotating}
\usepackage[normalem]{ulem}
\usepackage{amsmath}
\usepackage{textcomp}
\usepackage{marvosym}
\usepackage{wasysym}
\usepackage{amssymb}
\usepackage{capt-of}
\usepackage[hidelinks]{hyperref}
\tolerance=1000
\usepackage[utf8]{inputenc}
\usepackage{commath}
\usepackage{pgf}
\usepackage{tikz}
\usetikzlibrary{shapes, arrows}
\usepackage{marginnote}
\usepackage{listings}
\usepackage{enumerate}
\usepackage{algpseudocode}
\usepackage{algorithm}
\usepackage{mathtools}
\setlength{\parskip}{16pt plus 2pt minus 2pt}
\renewcommand{\arraystretch}{1.6}
\author{Oleg Sivokon}
\date{\textit{<2016-05-25 Wed>}}
\title{Assignment 16, Data-Structures}
\hypersetup{
  pdfkeywords={Data-Structures, Algorithms, Assignment},
  pdfsubject={Third assignment in the course Data-Structures},
  pdfcreator={Emacs 25.1.50.2 (Org mode 8.2.10)}}
\begin{document}

\maketitle
\tableofcontents


\definecolor{codebg}{rgb}{0.96,0.99,0.8}
\definecolor{codestr}{rgb}{0.46,0.09,0.2}
\lstset{%
  backgroundcolor=\color{codebg},
  basicstyle=\ttfamily\scriptsize,
  breakatwhitespace=false,
  breaklines=false,
  captionpos=b,
  framexleftmargin=10pt,
  xleftmargin=10pt,
  framerule=0pt,
  frame=tb,
  keepspaces=true,
  keywordstyle=\color{blue},
  showspaces=false,
  showstringspaces=false,
  showtabs=false,
  stringstyle=\color{codestr},
  tabsize=2
}
\lstnewenvironment{maxima}{%
  \lstset{%
    backgroundcolor=\color{codebg},
    escapeinside={(*@}{@*)},
    aboveskip=20pt,
    captionpos=b,
    label=,
    caption=,
    showstringspaces=false,
    frame=single,
    framerule=0pt,
    basicstyle=\ttfamily\scriptsize,
    columns=fixed}}{}
}
\makeatletter
\newcommand{\verbatimfont}[1]{\renewcommand{\verbatim@font}{\ttfamily#1}}
\makeatother
\verbatimfont{\small}%
\clearpage

\section{Problems}
\label{sec-1}

\subsection{Problem 1}
\label{sec-1-1}
\begin{enumerate}
\item Given a hash-table with chaining of initial capacity $m$.  What is the
probability four elements inserted will end up in the same bucket?
\item Given a hash-table with open addressing and elements $k_1$, $k_2$ and
$k_3$ inserted in that order, what is the chance of performing three
checks when inserting the third element?
\item Given hash-table s.t. its density is $1-\frac{1}{\lg n}$.  Provided the
table uses open addressing, what is the expected time of failed search as
a function of $n$?
\end{enumerate}

\subsubsection{Answer 1}
\label{sec-1-1-1}
Our simplifying assumption is that we draw hashing functions at random from
a universe of hashing functions allows us to say that a probability of a key
being hashed to a slot in the table of $m$ slots is $\frac{1}{m}$.  Using
product law we can conclude that the probability of four keys being mapped
to the same slot is
$\frac{1}{m}\times\frac{1}{m}\times\frac{1}{m}=\frac{1}{m^3}$.

\subsubsection{Answer 2}
\label{sec-1-1-2}
Using the same simplifying assumption as before, we see that for the element
$k_3$ to be places only after three checks it has to first collide with
$k_1$ and then with $k_2$.  Using product law gives the probability of
$\frac{1}{m}\times\frac{1}{m}=\frac{1}{m^2}$.

\subsubsection{Answer 3}
\label{sec-1-1-3}
Recall that the average time needed for failed search in a hash table with
open addressing is $\frac{1}{1-\alpha}$.  Substituting $1-\frac{1}{\lg n}$
in place of $\alpha$ obtains:
\begin{align*}
  \frac{1}{1-\alpha} &= \frac{1}{1-\frac{1}{\lg n}} \\
  &= \frac{\lg n}{\lg n - 1}
\end{align*}

\subsection{Problem 2}
\label{sec-1-2}
Given a set of rational numbers $S$ and a rational number $z$,
\begin{enumerate}
\item write an algorithm that finds two distinct summands of $z$ with running
time $\Theta(n)$.
\item Same as in (2), but for four summands and time $\Theta(n^2)$.
\end{enumerate}

\subsubsection{Answer 4}
\label{sec-1-2-1}

\subsubsection{Answer 5}
\label{sec-1-2-2}

\subsection{Problem 3}
\label{sec-1-3}
Given a binary search tree with $n$ nodes there are $n + 1$ \emph{left} and
\emph{right} nil-pointers.  After performing the following on this tree: If
\texttt{left[z]} = \texttt{nil}, then \texttt{left[z]} = \texttt{tree-predecessor(z)}, and if \texttt{right[z]}
= \texttt{nil}, then \texttt{right[z]} = \texttt{tree-cussessor(z)}.  The tree built in this way
is called ``frying pan'' (WTF?), and the arcs are called ``threads''.
\begin{enumerate}
\item How can one distinguish between actual arcs and ``threads''?
\item Write procedures for inserting and removing elements from this tree.
\item What is the benefit of using ``threads''?
\end{enumerate}

\subsubsection{Answer 6}
\label{sec-1-3-1}
Search tree invariant implies that left pointer must point at a node with
a value less than the node holding the pointer, but predecessor would have
a value larger than the node holding the pointer.  The situation for right
node is symmetrical.

\subsubsection{Answer 7}
\label{sec-1-3-2}

\subsubsection{Answer 8}
\label{sec-1-3-3}
None what so ever.  Whoever wrote this question is a brainless moron, who
has no idea of how computers work.  He seems to believe that it matters
whether the node stores a null pointer or a pointer to some other node in
terms of amount of memory used, which is absolute bullshit.

\subsection{Problem 4}
\label{sec-1-4}
Given array $A[1\dots n]$ s.t. 
\begin{align*}
  A[1] > \dots > A[p] \\
  A[p + 1] > \dots > A[q] \\
  A[q + 1] > \dots > A[n] \\
  A[1] < A[q] \\
  A[p + 1] < A[n]
\end{align*}

insert it into binary tree.
\begin{enumerate}
\item What is the height of the resulting tree?
\item Erase $A[p+1]$ and insert it anew: how will the height and the shape of
the tree change?
\end{enumerate}

\subsubsection{Answer 9}
\label{sec-1-4-1}

\subsubsection{Answer 10}
\label{sec-1-4-2}
% Emacs 25.1.50.2 (Org mode 8.2.10)
\end{document}