% Created 2016-06-12 Sun 22:49
\documentclass[11pt]{article}
\usepackage[utf8]{inputenc}
\usepackage[T1]{fontenc}
\usepackage{fixltx2e}
\usepackage{graphicx}
\usepackage{longtable}
\usepackage{float}
\usepackage{wrapfig}
\usepackage{rotating}
\usepackage[normalem]{ulem}
\usepackage{amsmath}
\usepackage{textcomp}
\usepackage{marvosym}
\usepackage{wasysym}
\usepackage{amssymb}
\usepackage{capt-of}
\usepackage[hidelinks]{hyperref}
\tolerance=1000
\usepackage[utf8]{inputenc}
\usepackage{commath}
\usepackage{pgf}
\usepackage{tikz}
\usetikzlibrary{shapes, arrows}
\usepackage{marginnote}
\usepackage{listings}
\usepackage{enumerate}
\usepackage{algpseudocode}
\usepackage{algorithm}
\usepackage{mathtools}
\setlength{\parskip}{16pt plus 2pt minus 2pt}
\renewcommand{\arraystretch}{1.6}
\author{Oleg Sivokon}
\date{\textit{<2016-05-25 Wed>}}
\title{Assignment 16, Data-Structures}
\hypersetup{
  pdfkeywords={Data-Structures, Algorithms, Assignment},
  pdfsubject={Third assignment in the course Data-Structures},
  pdfcreator={Emacs 25.1.50.2 (Org mode 8.2.10)}}
\begin{document}

\maketitle
\tableofcontents


\definecolor{codebg}{rgb}{0.96,0.99,0.8}
\definecolor{codestr}{rgb}{0.46,0.09,0.2}
\lstset{%
  backgroundcolor=\color{codebg},
  basicstyle=\ttfamily\scriptsize,
  breakatwhitespace=false,
  breaklines=false,
  captionpos=b,
  framexleftmargin=10pt,
  xleftmargin=10pt,
  framerule=0pt,
  frame=tb,
  keepspaces=true,
  keywordstyle=\color{blue},
  showspaces=false,
  showstringspaces=false,
  showtabs=false,
  stringstyle=\color{codestr},
  tabsize=2
}
\lstnewenvironment{maxima}{%
  \lstset{%
    backgroundcolor=\color{codebg},
    escapeinside={(*@}{@*)},
    aboveskip=20pt,
    captionpos=b,
    label=,
    caption=,
    showstringspaces=false,
    frame=single,
    framerule=0pt,
    basicstyle=\ttfamily\scriptsize,
    columns=fixed}}{}
}
\makeatletter
\newcommand{\verbatimfont}[1]{\renewcommand{\verbatim@font}{\ttfamily#1}}
\makeatother
\verbatimfont{\small}%
\clearpage

\section{Problems}
\label{sec-1}

\subsection{Problem 1}
\label{sec-1-1}
\begin{enumerate}
\item Given a hash-table with chaining of initial capacity $m$.  What is the
probability four elements inserted will end up in the same bucket?
\item Given a hash-table with open addressing and elements $k_1$, $k_2$ and
$k_3$ inserted in that order, what is the chance of performing three
checks when inserting the third element?
\item Given hash-table s.t. its density is $1-\frac{1}{\lg n}$.  Provided the
table uses open addressing, what is the expected time of failed search as
a function of $n$?
\end{enumerate}

\subsubsection{Answer 1}
\label{sec-1-1-1}
Our simplifying assumption is that we draw hashing functions at random from
a universe of hashing functions allows us to say that a probability of a key
being hashed to a slot in the table of $m$ slots is $\frac{1}{m}$.  Using
product law we can conclude that the probability of four keys being mapped
to the same slot is
$\frac{1}{m}\times\frac{1}{m}\times\frac{1}{m}=\frac{1}{m^3}$.

\subsubsection{Answer 2}
\label{sec-1-1-2}
Using the same simplifying assumption as before, we see that for the element
$k_3$ to be places only after three checks it has to first collide with
$k_1$ and then with $k_2$.  Using product law gives the probability of
$\frac{1}{m}\times\frac{1}{m}=\frac{1}{m^2}$.

\subsubsection{Answer 3}
\label{sec-1-1-3}
Recall that the average time needed for failed search in a hash table with
open addressing is $\frac{1}{1-\alpha}$.  Substituting $1-\frac{1}{\lg n}$
in place of $\alpha$ obtains:
\begin{align*}
  \frac{1}{1-\alpha} &= \frac{1}{1-\frac{1}{\lg n}} \\
  &= \frac{\lg n}{\lg n - 1}
\end{align*}

\subsection{Problem 2}
\label{sec-1-2}
Given a set of rational numbers $S$ and a rational number $z$,
\begin{enumerate}
\item write an algorithm that finds two distinct summands of $z$ with running
time $\Theta(n)$.
\item Same as in (2), but for four summands and time $\Theta(n^2)$.
\end{enumerate}

\subsubsection{Answer 4}
\label{sec-1-2-1}
See the auxilary function \texttt{two\_summands\_of} in the following answer for
illustration.

The basic idea is to put all numbers from $S$ into a hash-table.  Then loop
over $k_i$ the keys of the resulting hash-table and look for a key $k_j$,
s.t.  $k_j = z - k_i$.  This lookup can be done in constant time while loop
over the hash-table will take linear time.  Thus the eintire alrogrithm will
complete in time linear in number size of $S$.

\subsubsection{Answer 5}
\label{sec-1-2-2}
The idea is to compute all possible sums of two distinct elements in $S$.
This can be accomplished in quadratic time.  Then we run modified
\texttt{two\_summands\_of} on the resulting sums.  We modify it to allow for the same
sum produced in different ways.  It is again easy to see that the auxilary
function \texttt{two\_summands\_of} runs only once and in linear time in the size of
its input, that is quadratic in size of $S$, and the preparation step runs
in quadratic time as well.  Thus overall complexity of \texttt{four\_summands\_of} is
quadratic in size of $S$.

\lstset{language=C,label= ,caption= ,numbers=none}
\begin{lstlisting}
list two_summands_of(int sum, chashtable summands) {
    iterator* it = (iterator*)make_hashtable_iterator(summands);
    do {
        pair kv = (pair)((printable*)it)->val;
        printable* a = kv->first;
        int s = *(int*)a->val;
        printable* b = (printable*)make_printable_int(sum - s);
        printable* found = chashtable_get(summands, b);
        if (found != NULL) {
            if (s * 2 == sum && ((list)kv->last)->cdr == NULL)
                continue;
            else if (s * 2 == sum)
                return cons(((list)kv->last)->car,
                            cons(((list)kv->last)->cdr->car, NULL));
            else return cons(kv->last, cons(found, NULL));
        }
    } while (next(it));
    return NULL;
}
\end{lstlisting}

\lstset{language=C,label= ,caption= ,numbers=none}
\begin{lstlisting}
list four_summands_of(int sum, array summands) {
    chashtable table = make_empty_int_chashtable();
    size_t i, j;
    for (i = 0; i < summands->length; i++) {
        printable* a = summands->elements[i];
        for (j = i + 1; j < summands->length; j++) {
            printable* b = summands->elements[j];
            printable* halfsum = printable_sum(a, b);
            pair elts = make_pair();
            elts->first = a;
            elts->last = b;
            list wrapper = cons((printable*)elts, NULL);
            printable* found = chashtable_get(table, halfsum);
            if (found != NULL)
                wrapper = cons((printable*)found, wrapper);
            chashtable_put(table, halfsum, (printable*)wrapper);
        }
    }
    list four = two_summands_of(sum, table);
    if (four != NULL) four = append(four->car, four->cdr->car);
    return four;
}
\end{lstlisting}

\emph{(Look for utility functions in \url{../lib})}

\subsection{Problem 3}
\label{sec-1-3}
Given a binary search tree with $n$ nodes there are $n + 1$ \emph{left} and
\emph{right} nil-pointers.  After performing the following on this tree: If
\texttt{left[z]} = \texttt{nil}, then \texttt{left[z]} = \texttt{tree-predecessor(z)}, and if \texttt{right[z]}
= \texttt{nil}, then \texttt{right[z]} = \texttt{tree-cussessor(z)}.  The tree built in this way
is called ``frying pan'' (WTF?), and the arcs are called ``threads''.
\begin{enumerate}
\item How can one distinguish between actual arcs and ``threads''?
\item Write procedures for inserting and removing elements from this tree.
\item What is the benefit of using ``threads''?
\end{enumerate}

\subsubsection{Answer 6}
\label{sec-1-3-1}
Search tree invariant implies that left pointer must point at a node with
a value less than the node holding the pointer, but predecessor would have
a value larger than the node holding the pointer.  The situation for right
node is symmetrical.

\subsubsection{Answer 7}
\label{sec-1-3-2}
It's the same procedure you would use with a regular binary search tree,
however instead of checking for \texttt{NULL} you would check if the value pointed
at is larger or greater, depending on side you are inserting the child at.

\subsubsection{Answer 8}
\label{sec-1-3-3}
No difference what so ever.  It doesn't matter whether the pointer points
as some other node or nowhere.  It makes it more difficult to write code
to work with such trees, but that's about it.

\subsection{Problem 4}
\label{sec-1-4}
Given array $A[1\dots n]$ s.t. 
\begin{align*}
  A[1] > \dots > A[p] \\
  A[p + 1] > \dots > A[q] \\
  A[q + 1] > \dots > A[n] \\
  A[1] < A[q] \\
  A[p + 1] < A[n]
\end{align*}

insert it into binary tree.
\begin{enumerate}
\item What is the height of the resulting tree?
\item Erase $A[p+1]$ and insert it anew: how will the height and the shape of
the tree change?
\end{enumerate}

\subsubsection{Answer 9}
\label{sec-1-4-1}

\subsubsection{Answer 10}
\label{sec-1-4-2}
% Emacs 25.1.50.2 (Org mode 8.2.10)
\end{document}